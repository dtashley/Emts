\chapter{\cinttwotitle{}}

\label{cint2}


%%%%%%%%%%%%%%%%%%%%%%%%%%%%%%%%%%%%%%%%%%%%%%%%%%%%%%%%%%%%%%%%%%%%%%%%%%%%%%%
%%%%%%%%%%%%%%%%%%%%%%%%%%%%%%%%%%%%%%%%%%%%%%%%%%%%%%%%%%%%%%%%%%%%%%%%%%%%%%%
%%%%%%%%%%%%%%%%%%%%%%%%%%%%%%%%%%%%%%%%%%%%%%%%%%%%%%%%%%%%%%%%%%%%%%%%%%%%%%%
\section{Overview of This Manual}
\label{cint2:siov0}

This manual describes \emph{\productname{}}, a customizable tool integration
framework.

\emph{\productname{}} is a customizable tool integration framework:
\begin{itemize}
      \item It allows the same tools, built from the same source code,
            to be used in a consistent way in the following forms:
            \begin{itemize}
                  \item Standalone console-mode programs.
                  \item As built-in functions in a scripting language, \emph{CLIKE}.
                  \item As programs with a GUI interface.
            \end{itemize}
      \item It provides a base product with substantial functionality that can be
            customized and extended through:
            \begin{itemize}
                \item The addition of functionality packaged as standalone conole-mode programs,
                      built-in functions in a scripting langauge, and panels in GUI
                      program.
                \item Customized opening graphics.
                \item Customized help and contact information.
          \end{itemize}
\end{itemize}


%%%%%%%%%%%%%%%%%%%%%%%%%%%%%%%%%%%%%%%%%%%%%%%%%%%%%%%%%%%%%%%%%%%%%%%%%%%%%%%
%%%%%%%%%%%%%%%%%%%%%%%%%%%%%%%%%%%%%%%%%%%%%%%%%%%%%%%%%%%%%%%%%%%%%%%%%%%%%%%
%%%%%%%%%%%%%%%%%%%%%%%%%%%%%%%%%%%%%%%%%%%%%%%%%%%%%%%%%%%%%%%%%%%%%%%%%%%%%%%
\section{Overview of \productname{}}
\label{cint2:siov1}

\emph{\productname{}} contains a core, called the \emph{core}
or the \emph{\productname{} core},
that is not designed to be divided.  Any proprietary or custom tool built
using \emph{\productname{}} would contain the entire core, combined
with additional proprietary or custom content.

Outside of the core, the fundamental building block of \emph{\productname{}}
is the \emph{\productname{} module}, or \emph{module}.  A module is the
smallest unit that can be included or not included in a build
of \emph{\productname{}}.  A module is atomic and not designed to
divided.  A \emph{tool} generally corresponds to closely related functionality;
cryptographic hashing functions, for example.

A module may contain one or more \emph{tools}.  A tool generally
corresponds to narrow functionality; the SHA256 hash, for example.

A tool may contain one or more \emph{commands}.  A command
generally has very narrow scope to support a tool.  For example,
an SHA256 tool might contain two commands, one to calculate the
hash of a string, and another to calculate the hash of a file.

The notions of module, tool, and command are subjective enough
that no guarantees can be made about how they might be
defined.  The only guarantee that can be made is that a
module corresponds to an integral number of panels
in the graphical tool.


%%%%%%%%%%%%%%%%%%%%%%%%%%%%%%%%%%%%%%%%%%%%%%%%%%%%%%%%%%%%%%%%%%%%%%%%%%%%%%%
%%%%%%%%%%%%%%%%%%%%%%%%%%%%%%%%%%%%%%%%%%%%%%%%%%%%%%%%%%%%%%%%%%%%%%%%%%%%%%%
%%%%%%%%%%%%%%%%%%%%%%%%%%%%%%%%%%%%%%%%%%%%%%%%%%%%%%%%%%%%%%%%%%%%%%%%%%%%%%%
\section{Motivation for \emph{\productname{}}}
%Section tag: MFP0
\label{cint2:smfp0}

TBD.


%%%%%%%%%%%%%%%%%%%%%%%%%%%%%%%%%%%%%%%%%%%%%%%%%%%%%%%%%%%%%%%%%%%%%%%%%%%%%%%
%%%%%%%%%%%%%%%%%%%%%%%%%%%%%%%%%%%%%%%%%%%%%%%%%%%%%%%%%%%%%%%%%%%%%%%%%%%%%%%
%%%%%%%%%%%%%%%%%%%%%%%%%%%%%%%%%%%%%%%%%%%%%%%%%%%%%%%%%%%%%%%%%%%%%%%%%%%%%%%
\section{Detailed Description of \emph{\productname{}}}
%Section tag: DDP0
\label{cint2:sddp0}

This manual describes \emph{\productname{}}, a customizable tool integration
framework.

\emph{\productname{}} is a customizable tool integration framework:
\begin{itemize}
      \item It allows the same tools, built from the same source code,
            to be used in a consistent way in the following forms:
            \begin{itemize}
                  \item Standalone console-mode programs.
                  \item As built-in functions in a scripting language, \emph{CLIKE}.
                  \item As programs with a GUI interface.
            \end{itemize}
      \item It provides a base product with substantial functionality that can be
            customized and extended through:
            \begin{itemize}
                \item The addition of functionality packaged as standalone conole-mode programs,
                      built-in functions in a scripting langauge, and panels in GUI
                      program.
                \item Customized opening graphics.
                \item Customized help and contact information.
          \end{itemize}
\end{itemize}

%Certum card instructions.
%
%Cut the card out of the holder.
%
%Installed in the reader (took a little guessing to get it open).
%
%Installed the SmartCard reader drivers from the ACS website.
%
%Rebooted to be sure.
%
%Installed the ProCertum CardManager, 64-bit MSI.
%
%Rebooted per instructions.
%
%Read card, Initialized, then set my standard 6-digit value for both PIN and PUK (they are set identically).
%
%20240810:  Results from renewing code signing certificate.  Went through automatic verification
%           process tonight.  It involved a cellphone and taking a picture of my identity
%           document (passport), and some shots of my face.
%
%           Status of process unclear.  Believe I've done all I can do.  Should know by
%           Monday, which should be a work day for Certum.
%
%           Process is unclear.  I believed I would use old certificate to help authenticate
%           the renewal, but from the instructions it appears that I don't do this.
%
%           Could not find earlier notes anywhere, so it appears I will have to re-document
%           how to use the card, or search more for my earlier notes.
%
%
%
%
%
%
%
%
%
%
%
%
%
%
%
%
%
%
%
%
%
%
%
%
%
%
%
%
%
%
%End of file c_int2.tex
